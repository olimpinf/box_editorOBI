% To add an image or include a .tex file you need to add
% \CWD
% to the relative (to the main document) path.
%
% Example:
% \begin{figure}
%   \centering
%   \includegraphics{\CWD/images/example.pdf}
% \end{figure}

Coloque o seu enunciado (em português!) aqui, usando as macros
abaixo.

%
% For input, use one of the following
%
\inputdesc{The first line contains five integers $A_1$, $B_1$, $C_1$,
  $D_1$ and $E_1$ ($0 \leq A_1,B_1,C_1,D_1,E_1 \leq 1$), representing
  the connection points labeled \emph{a},\emph{b}, \emph{c}, \emph{d}
  and \emph{e} of the first connector in the pair. The second line
  contains five integers $A_2$, $B_2$, $C_2$, $D_2$ and $E_2$ ($0 \leq
  A_2,B_2,C_2,D_2,E_2 \leq 1$), representing the connection points
  labeled \emph{a},\emph{b}, \emph{c}, \emph{d} and \emph{e} of the
  second connector in the pair. In the input, a $0$ represents an outlet an
  a $1$ represents a plug.}

\inputdescline{an integer $N$ ($1 \le N \le 10^{100}$).}

%
% For output, use one of the following
%

\outputdesc{Output a line containing one character. 
If the connectors are compatible, then write the upper case letter `\texttt{Y}'; otherwise
write the uppercase letter `\texttt{N}'.}

\outputdescline{an integer}{the number of eleven-multiple-anagrams of $N$.
Because this number can be very large, you are required to output the remainder of dividing it by $10^9 + 7$.}

\subsection*{Restrições}
\begin{itemize}
\item coloque as restrições dos dados de entrada aqui.
\end{itemize}

\subsection*{Informações sobre a pontuação}
\begin{itemize}
\item Se aplicável, coloque as informações sobre a pontuação aqui; exemplo:
\item Para um conjunto de casos de testes valendo 20 pontos, $1 leq N \leq 10$.
\end{itemize}

%\sampleio will look for files named sample-n.in and sample-n.sol (where n is 1, 2, 3...)
%in the documents directory and include them as samples.

\sampleio
